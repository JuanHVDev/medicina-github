% Options for packages loaded elsewhere
\PassOptionsToPackage{unicode}{hyperref}
\PassOptionsToPackage{hyphens}{url}
%
\documentclass[
]{article}
\usepackage{lmodern}
\usepackage{amssymb,amsmath}
\usepackage{ifxetex,ifluatex}
\ifnum 0\ifxetex 1\fi\ifluatex 1\fi=0 % if pdftex
  \usepackage[T1]{fontenc}
  \usepackage[utf8]{inputenc}
  \usepackage{textcomp} % provide euro and other symbols
\else % if luatex or xetex
  \usepackage{unicode-math}
  \defaultfontfeatures{Scale=MatchLowercase}
  \defaultfontfeatures[\rmfamily]{Ligatures=TeX,Scale=1}
\fi
% Use upquote if available, for straight quotes in verbatim environments
\IfFileExists{upquote.sty}{\usepackage{upquote}}{}
\IfFileExists{microtype.sty}{% use microtype if available
  \usepackage[]{microtype}
  \UseMicrotypeSet[protrusion]{basicmath} % disable protrusion for tt fonts
}{}
\makeatletter
\@ifundefined{KOMAClassName}{% if non-KOMA class
  \IfFileExists{parskip.sty}{%
    \usepackage{parskip}
  }{% else
    \setlength{\parindent}{0pt}
    \setlength{\parskip}{6pt plus 2pt minus 1pt}}
}{% if KOMA class
  \KOMAoptions{parskip=half}}
\makeatother
\usepackage{xcolor}
\IfFileExists{xurl.sty}{\usepackage{xurl}}{} % add URL line breaks if available
\IfFileExists{bookmark.sty}{\usepackage{bookmark}}{\usepackage{hyperref}}
\hypersetup{
  hidelinks,
  pdfcreator={LaTeX via pandoc}}
\urlstyle{same} % disable monospaced font for URLs
\usepackage{longtable,booktabs}
% Correct order of tables after \paragraph or \subparagraph
\usepackage{etoolbox}
\makeatletter
\patchcmd\longtable{\par}{\if@noskipsec\mbox{}\fi\par}{}{}
\makeatother
% Allow footnotes in longtable head/foot
\IfFileExists{footnotehyper.sty}{\usepackage{footnotehyper}}{\usepackage{footnote}}
\makesavenoteenv{longtable}
\setlength{\emergencystretch}{3em} % prevent overfull lines
\providecommand{\tightlist}{%
  \setlength{\itemsep}{0pt}\setlength{\parskip}{0pt}}
\setcounter{secnumdepth}{-\maxdimen} % remove section numbering

\author{}
\date{}

\begin{document}

\hypertarget{ginecologuxeda-y-obstetricia}{%
\subsection{\# Ginecología y
Obstetricia}\label{ginecologuxeda-y-obstetricia}}

\hypertarget{que-es-la-ginecologuxeda}{%
\subsection{\#\# ¿Que es la
Ginecología?}\label{que-es-la-ginecologuxeda}}

Estudia la salud integral de la mujer y su sist reproductivoelectrónico

\begin{longtable}[]{@{}ll@{}}
\toprule
\begin{minipage}[b]{0.35\columnwidth}\raggedright
Ginecologia\strut
\end{minipage} & \begin{minipage}[b]{0.59\columnwidth}\raggedright
Obstetricia\strut
\end{minipage}\tabularnewline
\midrule
\endhead
\begin{minipage}[t]{0.35\columnwidth}\raggedright
Salud Integral de la mujer\strut
\end{minipage} & \begin{minipage}[t]{0.59\columnwidth}\raggedright
Atencion a las mujeres embarazadas\strut
\end{minipage}\tabularnewline
\begin{minipage}[t]{0.35\columnwidth}\raggedright
Mujeres con problema de salud reproductiva\strut
\end{minipage} & \begin{minipage}[t]{0.59\columnwidth}\raggedright
Atiende la preconcepcion, cuidados del embarazo y posibles
complicaciones\strut
\end{minipage}\tabularnewline
\begin{minipage}[t]{0.35\columnwidth}\raggedright
Se acude anualmente a chequeo\strut
\end{minipage} & \begin{minipage}[t]{0.59\columnwidth}\raggedright
Seguimiento del embarazo\strut
\end{minipage}\tabularnewline
\begin{minipage}[t]{0.35\columnwidth}\raggedright
Emplea tecnicas quirurgicas y no quirurgicas\strut
\end{minipage} & \begin{minipage}[t]{0.59\columnwidth}\raggedright
Consultas medicas durante el embarazo\strut
\end{minipage}\tabularnewline
\bottomrule
\end{longtable}

\hypertarget{violencia-gineco-obstetra}{%
\subsection{\#\#\# Violencia
Gineco-Obstetra:}\label{violencia-gineco-obstetra}}

{[}{[}Pasted image 20221019081504.png{]}{]}

\hypertarget{relacion-medico-paciente}{%
\subsection{\#\#\# Relacion
medico-paciente}\label{relacion-medico-paciente}}

Es una modalidad de las multiples relaciones interpersonales, con el fin
de que el enfermeo satisface su deseo y necesidad de salud y al medico
cumplir sus funciones sociales mas importantes. \#\#\# Estructura de la
relacion medico paciente: --- - Fin propio de la relacion; - La salud
del paciente - El modo propio de la relacion; - Equilibrida combinacion
de las operaciones objetivas y empaticas para el diagnostico y
tratamiento - El vinculo propio de la relacion; - Es aquel en el que
adquieren una realidad concreta los actos objetivantes y empaticos antes
mencionados. - Comunicacion medico-paciente; - Es el conjunto de los
recursos tecnicos, entre los cuales figura principalmente la palabra.
\#\#\# Momentos de la relacion medico-paciente: ---

{[}{[}Momentos-relacion-medico-paciente.svg{]}{]} \#\#\# Principios ---
{[}{[}Principios-medico-paciente.svg\textbar700x400{]}{]} \#\#\#
Terminologia medica en ginecología y obstetricia. --- \textbar{} Mujer
\textbar{} Gineco \textbar{} \textbar{} ---------------------------
\textbar{}
------------------------------------------------------------------------------------------------------------------------------------------------------------------------
\textbar{} \textbar{} Utero \textbar{} Histero \textbar{} \textbar{}
Seno \textbar{} Masto \textbar{} \textbar{} Mes \textbar{} Meno
\textbar{} \textbar{} Mensual \textbar{} Menstruo \textbar{} \textbar{}
Saco amniotico \textbar{} Amnio \textbar{} \textbar{} Puncion \textbar{}
Centesis \textbar{} \textbar{} Coito doloroso \textbar{} Dispareunia
\textbar{} \textbar{} Blanco \textbar{} Leuco \textbar{} \textbar{}
Menarca \textbar{} Primera menstruacion \textbar{} \textbar{} Perine
\textbar{} Episio \textbar{} \textbar{} Comienzo \textbar{} Arqui
\textbar{} \textbar{} Cuello \textbar{} Cervic \textbar{} \textbar{}
Vagina \textbar{} Colpo \textbar{} \textbar{} Ovario \textbar{} Oforo
\textbar{} \textbar{} Trompa de falopio \textbar{} Salpingo \textbar{}
\textbar{} Ovulo \textbar{} Ovul \textbar{} \textbar{} Detener
\textbar{} Paus \textbar{} \textbar{} Paralelo \textbar{} Para
\textbar{} \textbar{} Flujo \textbar{} Rrea \textbar{} \textbar{}
Fecundacion \textbar{} Union de gametos \textbar{} \textbar{} Embrion
\textbar{} Hasta la 8 semanas \textbar{} \textbar{} Feto \textbar{} De
la semana 9 al parto \textbar{} \textbar{} Implantacion \textbar{}
Proceso por el cual el blastocisto se fija al endometrio \textbar{}
\textbar{} Loquios \textbar{} Sangrado despues del parto \textbar{}
\textbar{} Entuertos \textbar{} Contracciones postparto para involucion
uterina \textbar{} \textbar{} Puerperio \textbar{} Del parto hasta 45
dias despues \textbar{} \textbar{} Perinatal \textbar{} Todo lo
referente al tiempo de gestacion \textbar{} \textbar{} Prenatal
\textbar{} Todo el tiempo antes de la gestacion \textbar{} \textbar{}
Recien nacido o neonato \textbar{} Primer mes de visa \textbar{}
\textbar{} Lactante \textbar{} Despues del primer mes o de 12-24 meses o
menor 28 dias a 12 meses \textbar{} \textbar{} Mortinato \textbar{} Nace
muerto \textbar{} \textbar{} Cloasma o melasma \textbar{}
Hiperpigmentacion \textbar{} \textbar{} Distocia \textbar{} Trabajo de
parto con problemas \textbar{} \textbar{} Eutosia \textbar{} Parto sin
roblema \textbar{} \textbar{} Gestante \textbar{} Mujer embarazada
\textbar{} \textbar{} Paridad \textbar{} Numero de partos \textbar{}
\textbar{} Nulipara o primipara \textbar{} No ha dado a luz a ninguno
hijo \textbar{} \textbar{} Multipara \textbar{} Mas de 5 veces
\textbar{} \textbar{} Granmultipara \textbar{} Mas de 8 veces \textbar{}
\textbar{} Involucion uterina \textbar{} Vuelta al tamaño original del
utero despues del parto \textbar{} \textbar{} Atonia uterina \textbar{}
Perdida del tono muscular del utero, retarda su involucion \textbar{}
\textbar{} Pelvimetria \textbar{} Medicion de los diametros pelvicos
\textbar{} \textbar{} Desproporcion cefalopelvica \textbar{} Pelvis
estrecha, cabeza del bebe muy grande \textbar{} \textbar{} Episiotomia y
episiorrafia \textbar{} Ampliacion del canal vaginal \textbar{}
\textbar{} Meconio \textbar{} Mezcla de liquido amniotico, moco, bilis,
celulas muertas y lanugo digerido por el feto \textbar{} \textbar{}
Hipoxia neonatal \textbar{} Disminucion en el aporte de oxigeno al
neonato \textbar{} \textbar{} Calostro \textbar{} Primera leche rica en
carotenos y anticuerpos \textbar{} \textbar{} Parto a termino \textbar{}
37 a 40 SDG \textbar{} \textbar{} Parto a pretermino \textbar{} Inferior
a 37 pero superioor a 20 SDG \textbar{} \textbar{} Aborto \textbar{}
Antes de las 20 SDG \textbar{} \textbar{} Embarao ectopico \textbar{}
Implantacion fuera del utero \textbar{} \textbar{} Macrosomia \textbar{}
Mas de 4000 g \textbar{} \textbar{} Preclampsia \textbar{} Es la presiòn
arterial alta y signos de daño hepàtico o renal que ocurren en las
mujeres después de la semana 20 de embarazo. \textbar{} \textbar{}
Amenorrea \textbar{} Ausencia de la menstruacion \textbar{} \textbar{}
Dismenorrea \textbar{} Dolor en la menstruacion \textbar{} \textbar{}
Polimenorrea \textbar{} Intervalo ente la menstruacion menor a 21 dias
\textbar{} \textbar{} Oligomenorrea \textbar{} Menstruacion ineficiente
\textbar{} \textbar{} Metrorragia \textbar{} Hemorragia Menstrual
\textbar{} \textbar{} Menorragia \textbar{} Sangrado excesivo \textbar{}
\textbar{} Eclampia \textbar{} Presencia de convulsiones generalizadas
en pacientes despues de la preclampsia \textbar{} \textbar{} Bridas
amnioticas \textbar{} Fusiones del amnios con la piel del feto con el
consiguiente riesgo de ``deformacion de muñones'' \textbar{} \textbar{}
Climaterio \textbar{} Periodo de transicion en la vida de la mujer,
entre la etapa reproductiva a la no reproductiv que inicia 5 años antes
de la menopausia y tiene una duracion de 5-10 años \textbar{} \textbar{}
Corioamnionitis \textbar{} Infeccion amniotica de la placenta y
membranas \textbar{}

\hypertarget{historia-clinica}{%
\subsection{\# Historia Clinica}\label{historia-clinica}}

\begin{itemize}
\tightlist
\item
  Principal herramienta diagnostica de un medico
\item
  Se debe de realizar de manera estruccturada y conocimientos adecuados.
  \#\# ¿Como debe realizarse? --- Debe de ser:
\item
  Sistematica
\item
  Estructurada
\item
  Digerida En Pacientes asintomaticas, embarazadas y sintomaticas.
\end{itemize}

\hypertarget{antecedentes-gineco-obstericos}{%
\subsection{\#\# Antecedentes
Gineco-Obstericos}\label{antecedentes-gineco-obstericos}}

\hypertarget{interrogatorio}{%
\subsection{\#\#\# Interrogatorio}\label{interrogatorio}}

\begin{itemize}
\tightlist
\item
  Menarquia(Edad de la primera menstruacion espontanea)Ocure entre los
  11-15 años de edad \#\#\#\# Caracteristicas de la menstrucaion ---
\item
  Duracion y cantidad de sangre: Normalmente duran de 2-6 dias
\item
  La cantidad la evalua la mujer segun lo que haya sido su experiencia
\item
  Frecuencia: Normal cada 25-28 dias
\item
  Fecha de la ultima menstruacion: Sirve para determinar posibilidades
  de embrazao, momento de ovulacion, toma de muestras para examenes
  hormonolaes
\item
  Menopausia: 45-55 años de edad \#\#\# Terminos usados respecto a las
  menstruaciones: ---
\item
  Dismenorrea: Menstruaciones dolorosas
\item
  Hipermenorrea o menorrragia: Menstruaciones abundantes
\item
  Hipomenorrea: Menstruaciones escasas
\item
  Polimenorrea: Intervalos menores de 21 dias
\item
  Oligomenorrea: Intervalos entre 36-90 dias
\item
  Amenorrea: Si no ocurren menstruaciones en 90 dias
\item
  Metrorragia: Si la hemorragia genital no se ajusta al ciclo sexual
  ovarico
\end{itemize}

\hypertarget{informacion-sobre-los-embarazos}{%
\subsection{\#\#\# Informacion sobre los
embarazos}\label{informacion-sobre-los-embarazos}}

\begin{itemize}
\tightlist
\item
  ¿Cuantos embarazos ocurrieron?
\item
  Si fueron de termino o prematuros
\item
  Si los partos fueron vaginales o por cesarea
\item
  Problemas asociados al embarazo(hipertension arterial, hiperglicemia,
  muerte fetal)
\item
  Antecedentes de abortos(espontaneos o provocados)
\item
  Numero de hijos vivos Los embarazos duran 40 semanas(9 meses) con
  variaciones entre 37 y 42 semanas, definiendose como: Parto de
  termino: Ocurre pasadas las 37 semanas de embarazo. Parto de
  pretermino o prematuro: Ocurre entre las 22 y 33 semanas. El recien
  nacido pesa menos de 2.500 gramos Aborto: Expulsion del feto antes de
  las 22 semanas, habitualmente presenta un peso menor de 500 gramos
  \#\# Otros datos que pueden ser de interes: --- \#\#\# Metodos
  anticonceptivos ---
\item
  Abstinencia en periodos fertiles
\item
  Anticonceptivos orales
\item
  DIU
\item
  Condon o preservativo, etc
\item
  Numero de parejas sexuales de alto o bajo riesgo
\item
  Presencia de otros flujos vaginales
\item
  Fecha de ultimo frotis cervical(Papanicolau) o de la ultima mamografia
\item
  Enfermedades o procedimientos ginecologicos(endometritis, anexitis,
  ETS, histerectomia) \#\# Antecedentes Perinatales --- Se necesita
  describir el embarazo y los eventos clinicos que acompañan este
  proceso desde la etapa preconcepcional como:
\item
  Edad de la madre
\item
  Estado nutricional
\item
  Controles prenatales
\item
  Patologias acompañadas al embarazo
\item
  Edad Gestacional
\item
  Peso
\item
  Talla
\item
  Necesidad de reanimacion u oxigeno al momento de nacer
\item
  Hospitalizaciones
\item
  Tiempo de duracion
\item
  Analisis del peso para la edad gestacional
\end{itemize}

\hypertarget{exploracion-fisica}{%
\subsection{\#\# Exploracion Fisica:}\label{exploracion-fisica}}

Consejos para la exploracion pelvica correcta

\begin{longtable}[]{@{}ll@{}}
\toprule
Paciente &\tabularnewline
\midrule
\endhead
Evitar relaciones sexuales durante 24-48h &\tabularnewline
Evitar duchas vaginales durante 24-48h &\tabularnewline
Evitar emplear supositorios vaginales durante 24-48h &\tabularnewline
Vaciar la vejiga antes de la exploracion &\tabularnewline
\bottomrule
\end{longtable}

\begin{verbatim}
title: Posicion de la Paciente
Colocarse en decubito supino, con la cabeza y los hombros elevados, con los brazos a los costados o cruzados sobre el torax para mejorar el contacto visual y reducir la tension de los musculos abdominales
\end{verbatim}

\begin{longtable}[]{@{}l@{}}
\toprule
\begin{minipage}[b]{0.97\columnwidth}\raggedright
Examinador\strut
\end{minipage}\tabularnewline
\midrule
\endhead
\begin{minipage}[t]{0.97\columnwidth}\raggedright
Obtener el permiso\strut
\end{minipage}\tabularnewline
\begin{minipage}[t]{0.97\columnwidth}\raggedright
Que haya un acompañante\strut
\end{minipage}\tabularnewline
\begin{minipage}[t]{0.97\columnwidth}\raggedright
Explicar cada paso de la exploracion\strut
\end{minipage}\tabularnewline
\begin{minipage}[t]{0.97\columnwidth}\raggedright
Cubrir a la paciente desde la mitad del abdomen hasta las rodillas\strut
\end{minipage}\tabularnewline
\begin{minipage}[t]{0.97\columnwidth}\raggedright
Hundir el paño entre las rodillas para establecer contacto visual con
ella\strut
\end{minipage}\tabularnewline
\begin{minipage}[t]{0.97\columnwidth}\raggedright
Evitar los movimientos bruscos o inesperados\strut
\end{minipage}\tabularnewline
\begin{minipage}[t]{0.97\columnwidth}\raggedright
Elegir un especulo del tamaño adecuado\strut
\end{minipage}\tabularnewline
\begin{minipage}[t]{0.97\columnwidth}\raggedright
Calentar el especulo con agua corriente\strut
\end{minipage}\tabularnewline
\begin{minipage}[t]{0.97\columnwidth}\raggedright
Vigilar la cara de la paciente durante la exploracion para verificar que
se encuentre comoda\strut
\end{minipage}\tabularnewline
\begin{minipage}[t]{0.97\columnwidth}\raggedright
Utilizar una tecnica excelente, pero delicada sobre todo cuando se
inserta el especulo\strut
\end{minipage}\tabularnewline
\bottomrule
\end{longtable}

\hypertarget{seleccion-de-equipo}{%
\subsection{\#\#\# Seleccion de Equipo}\label{seleccion-de-equipo}}

\begin{itemize}
\tightlist
\item
  Una fuente movil de luz adecuada
\item
  Un especulo vaginal de tamaño apropiado
\item
  Lubricante hidrosoluble
\item
  Equipo para la toma de muestras para la citologia vaginal, cultivos,
  sondas de ADN u otros materiales.
\end{itemize}

{[}{[}Pasted image 20221019085534.png{]}{]} \#\#\# Exploracion Externa
--- \textbar{} Parte \textbar{} Esperado \textbar{} \textbar{}
---------------- \textbar{}
------------------------------------------------------------------------------------------------------------------------------
\textbar{} \textbar{} Monte de venus \textbar{} Piel lisa y limpia
\textbar{} \textbar{} Vello pubico \textbar{} Distribuido de manera
uniforme \textbar{} \textbar{} Labios mayores \textbar{} Grandes o
cerrados, secos o humedos, arrugados o tersos, tejido blanco y
homogeneo, generalmente simetricos \textbar{} \textbar{} Labios menores
\textbar{} Humedos, con superficie interna de color rosado oscuro.
Tejido blando y homogeneo y sin sensibilidad dolorosa. \textbar{}
\textbar{} Clitoris \textbar{} Longitud de 2 cm o menor; diametro de 0.5
cm. Sin inflamacion o adherencias \textbar{} \textbar{} Orificio uretral
\textbar{} Aparecen como hendiduras o abertura irregular, cercana o al
introito vaginal o contenida en el, generalmente en la linea media
\textbar{} \textbar{} Orificio vaginal \textbar{} Aparece como hendidura
vertical fina u orificio amplio con bordes irregulares. Humedad tisular
\textbar{} \textbar{} Perine \textbar{} Superficie del perine lisa:
generalmete gruesa y uniforme en la mujer nulipara, mas delgada y rigida
en la mujer multipara \textbar{} \textbar{} Ano \textbar{} Piel oscura
pigmentanda y posiblemente engrosada \textbar{}

\begin{longtable}[]{@{}ll@{}}
\toprule
\begin{minipage}[b]{0.14\columnwidth}\raggedright
Parte\strut
\end{minipage} & \begin{minipage}[b]{0.80\columnwidth}\raggedright
Inesperado\strut
\end{minipage}\tabularnewline
\midrule
\endhead
\begin{minipage}[t]{0.14\columnwidth}\raggedright
Glandulas de Skene\strut
\end{minipage} & \begin{minipage}[t]{0.80\columnwidth}\raggedright
Secrecion o sensibilidad dolorosa, observe el olor, la consistencia, el
olor de las posibles secreciones; obtenga cultivo\strut
\end{minipage}\tabularnewline
\begin{minipage}[t]{0.14\columnwidth}\raggedright
Glandulas de Bartolino\strut
\end{minipage} & \begin{minipage}[t]{0.80\columnwidth}\raggedright
Inflamacion, sensibilidad dolorosa, masas, calor, fluctuacion o
secrecion. Observar el color, consistencia, olor de cualquier posible
secrecion\strut
\end{minipage}\tabularnewline
\bottomrule
\end{longtable}

\hypertarget{exploracion-interna}{%
\subsection{\#\#\# Exploracion interna}\label{exploracion-interna}}

\hypertarget{cuello-uterino}{%
\subsection{\#\#\#\# Cuello Uterino}\label{cuello-uterino}}

Color: Rosado con distribucion uniforme Posicion: En la linea media
horizontal u orientado anterior o posteriormente, protruyendo en la
vagina de 1-3 cm. el cuello uterino en edad gestaciones tiene diametro
de 2-3 cm Caracteristicas superficiales: Superficie lisa Forma: Uniforme
Tamaño: 3 cm de diametro Secrecion: Sin olor, de textura pastosa o
liquida transparente, densa, fina o fibrosa(con frecuencia mas densa en
la fase central del ciclo menstrual o inmediatamente antes de la
menstruacion) Tamaño y forma del orificio cervical: Mujer
nulipara:Pequeña, redondeada y oval Mujer multipara: Generalmente, en
forma de hendidura horizontal, irregular o estrellada

Al retirar el especulo\ldots{}

\begin{longtable}[]{@{}ll@{}}
\toprule
\begin{minipage}[b]{0.64\columnwidth}\raggedright
Partes\strut
\end{minipage} & \begin{minipage}[b]{0.30\columnwidth}\raggedright
Esperado\strut
\end{minipage}\tabularnewline
\midrule
\endhead
\begin{minipage}[t]{0.64\columnwidth}\raggedright
Paredes vaginales\strut
\end{minipage} & \begin{minipage}[t]{0.30\columnwidth}\raggedright
Color rosado, similar al del cuello o mas claro, pared humeda, lisa o
rugosa, y homogenea. Secreciones sin olor, no densas, claras o
turbias\strut
\end{minipage}\tabularnewline
\bottomrule
\end{longtable}

Exploracion Bimanual:

\begin{longtable}[]{@{}ll@{}}
\toprule
\begin{minipage}[b]{0.05\columnwidth}\raggedright
Paredes\strut
\end{minipage} & \begin{minipage}[b]{0.90\columnwidth}\raggedright
Esperados\strut
\end{minipage}\tabularnewline
\midrule
\endhead
\begin{minipage}[t]{0.05\columnwidth}\raggedright
Paredes vaginales\strut
\end{minipage} & \begin{minipage}[t]{0.90\columnwidth}\raggedright
Lisa, Homogenea y sin sensibilidad dolorosa\strut
\end{minipage}\tabularnewline
\begin{minipage}[t]{0.05\columnwidth}\raggedright
Cuello uterino\strut
\end{minipage} & \begin{minipage}[t]{0.90\columnwidth}\raggedright
Tamaño, forma, longitud: Coherentes con los de la exploracion con el
especulo, Consistencia: Firme en mujeres no gestantes, mas blanda en
mujeres gestantes, Posicion: En la linea media horizontal u orientado
anterior y posteriormente, protruyendo en la vagina de 1-3cm. Movilidad:
Movimiento de 1-2 cm en cada direccion, minimas molestias\strut
\end{minipage}\tabularnewline
\begin{minipage}[t]{0.05\columnwidth}\raggedright
Utero\strut
\end{minipage} & \begin{minipage}[t]{0.90\columnwidth}\raggedright
Localizacion y posicion: En la linea media horizontal u orientado
anterior o posteriormente, protruyendo en la vagina de 1-3cm. Tamaño,
forma, contorno: De 5.5-8cm de largo, forma de pera, contorno redondeado
y en mujeres no gestantes; paredes firmes y lisas. Movilidad: movil en
el plano anteroposterior\strut
\end{minipage}\tabularnewline
\begin{minipage}[t]{0.05\columnwidth}\raggedright
Ovarios\strut
\end{minipage} & \begin{minipage}[t]{0.90\columnwidth}\raggedright
Consistencia: Si son palpables, los obarios deben notarse firmes, tensos
y leve o moderadamente sensibles. Tamaño y forma: En torno a 3x2x1cm
Ovoide\strut
\end{minipage}\tabularnewline
\bottomrule
\end{longtable}

\hypertarget{medicion-de-altura-uterina}{%
\subsubsection{Medicion de altura
uterina:}\label{medicion-de-altura-uterina}}

\textbf{Altura uterina:} Va del borde superior de la sinfisis pubica al
fondo uterino, circunferencia abdominañ

{[}{[}Pasted image 20221019091303.png{]}{]}

{[}{[}Pasted image 20221019091456.png{]}{]}

\hypertarget{auscultacion-de-latidos-cardiacos-fetales}{%
\subsubsection{Auscultacion de latidos cardiacos
fetales}\label{auscultacion-de-latidos-cardiacos-fetales}}

Se realiza por medio de doopler obstetrico(mayor a 12 semanas) y por
medio del estetoscopio de Pinard(mayores de 20 semanas)

\hypertarget{doopler-obstetrico}{%
\subsubsection{Doopler obstetrico:}\label{doopler-obstetrico}}

\begin{enumerate}
\def\labelenumi{\arabic{enumi}.}
\tightlist
\item
  Se aplica una gota de gel para ultrasonido en el transductor doppler,
  se apoya el transductor sobre el abdomen materno (en el foco
  apropiado) presionando suavemente para evitar ruido
\item
  Se enciende el aparato y se ajusta el volumen, el transductor se
  inclina lentamente en varias direcciones hasta oír los latidos.
\item
  No es aconsejable usar una gran cantidad de gel, o mover el
  transductor rápidamente sobre el abdomen.
\end{enumerate}

\hypertarget{estetoscopio-de-pinard}{%
\subsubsection{Estetoscopio de Pinard}\label{estetoscopio-de-pinard}}

\begin{enumerate}
\def\labelenumi{\arabic{enumi}.}
\tightlist
\item
  Se coloca la parte ancha del estetoscopio de Pinard en el foco de
  auscultación máxima, se aplica el pabellón auricular sobre el extremo
  opuesto del estetoscopio
\item
  Se va presionando suavemente y se quita la mano de modo que el
  estetoscopio quede aprisionado entre el abdomen materno y el pabellón
  auricular.
\item
  Con el silencio ambiental necesario es posible detectar los LCF y
  medir su frecuencia en un minuto. El rango normal es 110-160 latidos
  por minuto. \#\# Maniobras de Leopold Se utilizan para determinar el
  fondo uterino, posición, presentación y el grado de encajamiento
  fetal. Se debe realizar a partir de las 20- 22 semanas de gestación,
  se obtienen mejores resultados a la semana 32 \#\#\# Primera Maniobra:
  Diagnostico de presentacion fetal: identificar que polo fetal se
  encuentra ocupando el fondo uterino, puede ser el polo cefalico o el
  polo pelvico \#\#\# Segunda maniobra Diagnostico de posicion
  ysituacion fetal Posicion del feto ya sea izquierda o derecha y la
  situacion ya sea longitudinal o transversa fetal. Se define como el
  lado materno(derecho o izquierdo) en el que se encuentra en el dorso
  del producto. \#\#\# Tercera maniobra Diagnostico de presentacion
  fetal Identificar la altura de la presentacion(libre, abocado y
  encajado), y corroborar la presentacion. La definimos como la
  presentacion se encontrara encajada o no. \#\#\# Cuarta maniobra
  Diagnostico de actitud fetal y relacion con la pelvis Se identifica la
  presentacion y se corrobora la altura de la presentacion(libre,
  abocado y encajado). Define que tipo de presentacion se aboca al
  estrecho superior de la pelvis.
\end{enumerate}

\hypertarget{tacto-vaginal-obstetrico}{%
\subsection{Tacto vaginal obstetrico:}\label{tacto-vaginal-obstetrico}}

Es parte del examen fisico de la mujer embarazada; y permite obtener
informacion sobre el cuello uterino, el polo fetal y la pelvis materna
El tacto vaginal obstetrico no se efectua de modo rutinario durante el
control prenatal -\textgreater{} + Contracciones uterinas frecuentes
independiente de la edad gestacional. + Sangrado vaginal, habiendo
descartado una placenta previa. + Embarazo de termino, para estimar si
existe o no cercania al parto \#\# Evaluacion del cuello uterino
{[}{[}Pasted image 20221019092440.png{]}{]}

Situacion Presentacion Variedad de posicion altura de posicion amplitud

\hypertarget{diagnostico-de-embarazo}{%
\section{Diagnostico de embarazo}\label{diagnostico-de-embarazo}}

De sospecha - Amenerrea - nausea y vómito - Sialerrea - Pelaquiuria -
Mastalgia - Astenia y adinamia - Mareos - Irritabilidad - Alteración del
gusto y del olfato - Somnelencia - ``Antojo'' de alimentos y bebidas -
Aumento de tamaño consiste y sensibilidad mamaria - Pigmentación de
pezón - Arecla secundaria - Red venosa de Haller - Tubércules de
Montgomery - Calostro - Pigmentación cutánea en abdomen, muslos,
genitales externos - Leucorrea per meniliasis - Mediante los métodos de
gabinete De probabilidad - Intensificación de varios síntomas de
sospecha - Percepción de movimientes fetales de la paciente -
Modificaciones de órganos pélvico: - coloración violácea de vagina -
Cuerpo uterina globoso y fondos de saco ocupados - Irregularidad en el
cuello uterino - Aumento del tamaño uterino acorde con la amenorrea -
Pruebas de laberaterio positivas - Presencia de células naviculares -
Falta de cristalización del moco cervical De certeza - Auscultación de
latidos cardiacos fetales - Percepción de parte fetales por el medico -
Comprobación de movientes fetales en exploración - Percepción de
contracciones de Braxton-Hicks - Actividad cardiaca presente en
electrocardiograma fetal - Esqueleto fetal visibilidad a los rayos X -
Latidos cardiacos fetales audibles - Sombra fetal en ultrasonido

\hypertarget{adaptacion-amterna-en-el-embarazo}{%
\section{Adaptacion amterna en el
embarazo}\label{adaptacion-amterna-en-el-embarazo}}

El embarazo implica varios cambios en anatomía, fisiología y bioquímica
que pueden poner en riesgo las reservas del cuerpo de la madre. Es
esencial un conocimiento básico de estas adaptaciones para comprender
los resultados normales en pruebas de laboratorio, conocer los fármacos
que pueden requerir ajustes en dosis y reconocer a las mujeres
predispuestas a sufrir complicaciones médicas durante el embarazo.

\hypertarget{sistema-cardiovascular}{%
\subsection{SISTEMA CARDIOVASCULAR}\label{sistema-cardiovascular}}

\hypertarget{cambios-anatuxf3micos}{%
\subsubsection{Cambios anatómicos}\label{cambios-anatuxf3micos}}

Con el crecimiento del útero y la elevación del diafragma, el corazón
gira sobre su eje longitudinal con un desplazamiento hacia arriba y a la
izquierda. En general, el tamaño del corazón aumenta alrededor de 12\%,
lo cual produce un incremento tanto en la masa miocárdica como en el
volumen intracardiaco (cerca de 80 ml). \#\#\# Volumen sanguineo La
expansión del volumen de sangre comienza al principio del primer
trimestre, aumenta con rapidez en el segundo trimestre y alcanza una
meseta alrededor de la semana 30. El aumento en la producción de
estrógeno por parte de la placenta estimula al sistema
renina-angiotensina que, a su vez, provoca mayores concentraciones
circulantes de aldosterona, la cual promueve la reabsorción de Na+ y la
retención de agua. La somatomamotropina coriónica humana, la
progesterona y quizá otras hormonas, promueven la eritropoyesis, que
aumenta cerca de 30\% la masa eritrocitaria. Esta hipervolemia del
embarazo compensa la pérdida materna de sangre durante el parto, que
promedia 500-600 ml para el parto vaginal y 1 000 ml en un parto por
cesárea. \#\#\# Gasto cardiaco El gasto cardiaco aumenta alrededor de
40\% durante el embarazo, y los valores máximos se alcanzan a las 20-24
semanas de gestación. Se piensa que este aumento es resultado de los
cambios hormonales del embarazo, al igual que del efecto de un
cortocircuito arteriovenoso de la circulación uteroplacentaria.

\hypertarget{frecuencia-cardiaca}{%
\subsubsection{Frecuencia cardiaca}\label{frecuencia-cardiaca}}

Con el crecimiento La frecuencia cardiaca materna en reposo, que aumenta
de manera progresiva en el curso de la gestación, al término de ésta
promedia cerca de 15 latidos por minuto más en comparación con la
frecuencia de mujeres no embarazadas \#\#\# Presión arterial La presión
arterial sistémica disminuye un poco durante el embarazo y llega al
punto más bajo a las 24-28 semanas de gestación. La presión diferencial
se amplía debido a que la disminución es mayor para la presión
diastólica que para la sistólica. Las presiones sistólica y diastólica
(y la presión arterial media) aumenta a los niveles preembarazo cerca de
las 36 semanas. \#\#\# Resistencia vascular periférica La resistencia
vascular disminuye en el primer trimestre y llega a un punto mínimo de
casi 34\% por debajo de los niveles sin embarazo a las 14 a 20 semanas
de gestación, con un ligero aumento hacia el término. Es probable que
los cambios hormonales del embarazo activen esta disminución en la
resistencia vascular al aumentar los vasodilatadores locales, como el
óxido nítrico, prostaciclina y tal vez adenosina.

\hypertarget{sistema-pulmonar}{%
\subsection{SISTEMA PULMONAR}\label{sistema-pulmonar}}

\hypertarget{cambios-anatuxf3micos-1}{%
\subsubsection{Cambios anatómicos}\label{cambios-anatuxf3micos-1}}

El embarazo altera la circulación de varios tejidos que participan en la
respiración. Por ejemplo, la dilatación capilar conduce a congestión de
la nasofaringe, laringe, tráquea y bronquios. A medida que crece el
útero, el diafragma se eleva hasta 4 cm. La caja torácica se desplaza
hacia arriba, aumentando el ángulo de las costillas respecto a la
columna vertebral. Estos cambios aumentan el diámetro torácico inferior
alrededor de 2 cm, y la circunferencia del tórax hasta en 6 cm. \#\#\#
Volúmenes y capacidades pulmonares El volumen de espacio muerto aumenta
debido a la relajación de la musculatura de las vías de conducción. El
volumen corriente y la capacidad inspiratoria aumentan. La elevación del
diafragma se asocia con reducción en la capacidad pulmonar total y en la
capacidad residual funcional.

\hypertarget{respiraciuxf3n}{%
\subsubsection{Respiración}\label{respiraciuxf3n}}

El aumento en las concentraciones de progesterona parece tener una
función esencial en la hiperventilación del embarazo, que se desarrolla
desde el principio del primer trimestre. Esta hiperventilación, que
disminuye la PCO2 a cerca de 27-32 mm Hg. produce una leve alcalosis
respiratoria (pH sanguineo de 7.4-7.5). La hiperventilación y la
circulación hiperdinámica aumentan un poco la P02 arterial. \#\# SISTEMA
RENAL \#\#\# Función renal La perfusión renal elevada es el principal
factor que participa en el aumento en la tasa de filtración glomerular
(GFR), que incrementa en cerca de 25\% en la segunda semana después de
la concepción. La GFR llega a un aumento máximo de 40 a 65\% para el
final del primer trimestre y continúa elevada hasta el término. Aunque
la GFR aumente de manera notable durante la gestación, el volumen de
orina que se excreta por día permanece igual. La depuración renal de
creatinina aumenta a medida que se eleva la GFR, con depuraciones
máximas 50\% mayores que los niveles de mujeres no embarazades. El
aumento en la GFR con saturación de la capacidad de reabsorción tubular
de la glucosa filtrada puede conducir a glucosuria. Las concentraciones
de angiotensina Il también aumentan durante el embarazo, pero no ocurren
la vasoconstricción e hipertensión que podrían esperarse \#\#\# Cambios
anatómicos Durante el embarazo, aumenta la longitud de los riñones en 1
a 1.5 cm, con aumento proporcional en el peso. Los cálices y pelvis
renales se dilatan y el volumen de la pelvis renal aumenta hasta seis
veces en comparación con el valor de 10 ml cuando no existe embarazo.
Los uréteres se dilatan por arriba del borde de la pelvis ósea, con
efectos más prominentes del lado derecho, se alargan, amplian y se
vuelven más curvos. Es posible que todo el sistema colector dilatado
contenga hasta 200 ml de orina, lo cual predispone a infecciones
urinarias ascendentes.

\hypertarget{vejiga}{%
\subsubsection{Vejiga}\label{vejiga}}

A medida que crece el útero, la vejiga urinaria se desplaza hacia arriba
y se aplana en el diámetro anteroposterior. Uno de los primeros sintomas
del embarazo es el aumento en la frecuencia urinaria, que quizá se
relacione con las hormonas del embarazo.

\hypertarget{sistema-gastrointestinal}{%
\subsection{SISTEMA GASTROINTESTINAL}\label{sistema-gastrointestinal}}

\hypertarget{esuxf3fago-y-estuxf3mago}{%
\subsubsection{Esófago y estómago}\label{esuxf3fago-y-estuxf3mago}}

El embarazo se asocia con una mayor producción de gastrina, que aumenta
el volumen y la acidez de las secreciones gástricas. Asimismo, es
posible que aumente la producción gástrica de moco. Disminuye la
peristalsis esofágica. La mayoría de las mujeres reportan sintomas de
reflujo en el primer trimestre, aunque los sintomas pueden volverse más
graves con el avance de la gestación. La predisposición subyacente al
reflujo en el embarazo se relaciona con la relajación del esfinter
esofágico inferior por intermediación hormonal \#\#\# Cambios anatómicos
A medida que el útero crece, el estómago se desplaza hacia arriba y los
intestinos delgado y grueso se extienden hacia regiones más
rostrolaterales. \#\#\# Cavidad bucal Parece haber un incremento en la
salivación, aunque es posible que esto se deba en parte a la dificultad
de deglución que aparece con las náuseas. Es posible que las encías se
vuelvan hipertróficas e hiperémicas; con frecuencia también se vuelven
tan esponjosas y friables que sangran con facilidad.

\hypertarget{intestinos}{%
\subsubsection{Intestinos}\label{intestinos}}

Los tiempos de tránsito intestinal disminuyen en el segundo y tercer
trimestre. Se ha pensado que la reducción en la motilidad intestinal
durante el embarazo ocurre por el aumento en las concentraciones
circulantes de progesterona. El tránsito lento de los alimentos a través
de las vias gastrointestinales aumenta potencialmente la absorción de
agua, lo cual prédispone al estreñimiento.

\hypertarget{metabolismo}{%
\subsection{METABOLISMO}\label{metabolismo}}

El embarazo aumenta los requerimientos nutricionales y ocurren varias
alteraciones en la madre cuya finalidad es satisfacer esta demanda. En
general, el apetito y la ingesta de alimentos aumentan, aunque algunas
tienen disminución del apetito o sufren náuseas y vómito. Los cambios
físicos más obvios son el aumento de peso y la alteración de la figura
corporal. El aumento promedio de peso durante la gestación es de 12.5 kg
En la segunda mitad de la gestación hay un aumento de lípidos en plasma
(el colesterol aumenta 50\%, la concentración de triglicéridos puede
triplicarse) El embarazo se asocia con resistencia a la insulina, lo
cual puede conducir a hiperglucemia (diabetes gestacional) en mujeres
susceptibles. En general, este trastorno metabólico desaparece después
del parto, pero puede presentarse de nuevo más tarde como diabetes tipo
2

\hypertarget{vigilancia-materno-fetal}{%
\subsection{VIGILANCIA MATERNO FETAL}\label{vigilancia-materno-fetal}}

Es el proceso que consiste en la evaluación del bienestar y la salud
fetal durante la etapa del embarazo. • Son pruebas que, de forma
sistémica, detectan a los fetos que presentan una pérdida del bienestar
fetal, representando indirectamente un peligro dentro del útero, por
ello, se debe tomar medidas oportunas encaminadas a prevenir un daño
irreversible - Son técnicas y procedimientos clínicos, bioquímicos,
ecográficos y bioelectrónicos. Los métodos de vigilancia fetal más
utilizaos son: \#\#\# Método clinico: • La Historia Clinica • La
Medición de la altura uterina • La observación de los movimientos
fetales \#\#\# Método bioquímico: • Medición ácido-básico (Scalp-Fetal)
\#\#\# Método biofásico detecta: • El Monitoreo de la FC fetal o la
prueba sin estrés (NST ``non stress test'') - La prueba con estrés (OCT
``oxitocin contraction test'') • El Perfil Biofásico Fetal (PBF) es un
análisis prenatal que combina la frecuencia cardíaca fetal (prueba sin
enfuerzo) y la ecografia fetal para evlaluar la FC, la respiración, los
movimientos, el tono muscular y el nivel del liquido amniótico del feto.
\#\#\# Indicaciones \#\#\#\# De causa materna: - HTA crónia o severa -
DM - Nefropatia, IR,LES - Amenaza o antecedente de parto pretérmino -
Anemia crónica o aguda - Hemoglobinopatias - Neuropatias agudas o
crónicas - Enfermedades cardiovasculares congénitas o adquiridas. -
Medicmaentos, drogas - Endocrinopatias \#\#\#\# Indicaciones de causa
placentaria: - Restricción de crecimiento intrauterino - Insuficiencia
útero placentaria - Embarazo prolongado(mayor a 42 sem) - Oligo o
polihidramnios \#\#\#\# Indicaciones de causa fetales: - Distocia
funicular - Soplo funicular - Laterocidencia - Transfusión feto fetal -
Disminución de los movimientos fetales - Embarazo múltiple - Infección
fetal - Ruptura de membranas - Defectos congénitos prematura

\hypertarget{pruebas-utilizadas}{%
\subsection{Pruebas utilizadas}\label{pruebas-utilizadas}}

Los protocolos son efectivos en reducir la morbimortalidad perinatal y
las secuelas neurológicas a largo plazo. Utilidad: - Identificar el
compromiso fetal - Actuar antes de la hipoxia y acidosis metabólica.
\#\#\# Metodos de vigilancia: - Vigilancia fetal clinica - Vigilancia
fetal electrónica - Vigilancia fetal ecográfica - Vigilancia fetal
bioquímica

\hypertarget{vigilancia-fetal-cluxednica}{%
\paragraph{Vigilancia fetal clínica}\label{vigilancia-fetal-cluxednica}}

\begin{itemize}
\tightlist
\item
  Se basa en HC y en los controles prenatales adecuados y oportunos:
  identificando factores de riesgo y establecer posibles complicaciones
  maternas o fetales
\item
  Se determina el crecimiento uterino
\item
  Aumento de peso materno
\item
  Edad gestacional por Altura Uterina

  \begin{itemize}
  \tightlist
  \item
    Sobre sinfisis del pubis: 10 sem
  \item
    Punto medio entre sinfisis del pubis y ombligo: 12-14 sem
  \item
    A la altura del ombligo: 20-22 sem
  \item
    Debajo del borde costal: 36 sem
  \item
    Alcanza el apéndice xifoides del esternón hacia el termino del
    embarazo. REGLA DE MCDONALD: Altura de fondo uterino (cm) x 2 / 7 =
    Edad Gestacional en meses lunares \#\#\#\#\# Movimientos fetales
    (MF)
  \end{itemize}
\item
  La variación de MF es amplia y pueden llegar a ser entre 5-30 por
  hora, no debe ser \textless3 MF por hora en el tercer trimestre
\item
  Deben ser \textgreater10 MF en 12 horas
\item
  En multiparas: perciben los MF en las semanas 16-18
\item
  En nulípara: 20-22 semanas
\item
  Fácil aplicación y la madre puede anotar el n° de MF en una hora por
  la mañana o tarde o por 30 minutos 3 veces por dia. \#\#\#\#
  Vigilancia fetal: electrónica Se realiza mediante instrumentos
  electrónicos llamados cardiotocógrafos, detectando:
\item
  FCF
\item
  MF
\item
  Actividad uterina (contracciones)
\item
  Reacción frente a estímulos externos (sonoros, manual) Objetivo:
  detectar lo más precoz y seguro alguna alteración del bienestar fetal
  por algun grado de hipoxia, antes de que exista daño físico o
  neurológico irreversible. Son 2 pruebas:
\item
  NST test no estresante
\item
  CST teste estresante
\end{itemize}

\hypertarget{external-fetal-monitor}{%
\subparagraph{External Fetal Monitor}\label{external-fetal-monitor}}

Consiste básicamente en 2 dispositivos. - El 1ro mide la intensidad de
las contracciones uterinas en kilopascales - El 2do mide la FC del feto
en lpm. Ambos parámetros se registran en un papel

\hypertarget{quuxe9-analizamos}{%
\subparagraph{¿Qué analizamos?}\label{quuxe9-analizamos}}

\begin{itemize}
\tightlist
\item
  Contracciones uterinas

  \begin{itemize}
  \item
    La parte de abajo son contracciones uterinas cuadradito
  \item
    Cada cuadradito equivale a 10 Kpascales
  \item
    Cada cuadradito equivale a 30 segundos
  \item
    La parte superior representa la FC fetal que cambia cada segundo.
  \item
    Se estudian en 10 min
  \item
    Normal: 4-5 contracciones en 10 min
  \item
    \textless3 = HIPODINAMIA
  \item
    \begin{quote}
    5= TAQUISISTOLIA
    \end{quote}
  \end{itemize}
\item
  Frecuencia cardíaca basal

  \begin{itemize}
  \tightlist
  \item
    Es el promedio de la FCF
  \item
    Lo normal 110-160
  \item
    Se considera NO TRANQUILIZADOR si disminuye hasta 100 o aumenta
    hasta 180
  \item
    ANORMAL menos de 100 o más de 180
  \end{itemize}
\item
  Variabilidad de la FC

  \begin{itemize}
  \tightlist
  \item
    Se analiza en un minuto
  \item
    La FCF debe fluctuar entre 5 a 25 latidos
  \item
    Si varía menos de 5 se le llama PATRÓN SILENTE
  \item
    Tipos:

    \begin{itemize}
    \tightlist
    \item
      MINIMA: 0-5 LPM
    \item
      MODERADA: 5-10 LPM
    \item
      NORMAL: 10-25 LPM
    \item
      AUMENTADA: 25 A MAS
    \end{itemize}
  \item
    Normal: Entre 1 a 5 cuadraditos, es más de 5 latidos por minuto
  \item
    Anormal: no pasa 1 cuadradito patrón silente
  \end{itemize}
\item
  Aceleraciones

  \begin{itemize}
  \tightlist
  \item
    Son aumentos de la FCF de \textgreater15 Ipm en \textgreater15
    segundos.
  \item
    Son hallazgos normales
  \end{itemize}
\item
  Desaceleraciones

  \begin{itemize}
  \tightlist
  \item
    Son disminuciones de la FCF de \textgreater15 lpm en \textgreater15
    segundos
  \item
    Son hallazgos anormales
  \item
    DIPS TIPO 1 TEMPARANAS): se inician con la contracción y regresan a
    su valor basal al terminar esta. No suelen descender por debajo de
    100 LPM. Se presentan con mas frecuencia cuando hay bolsas rotas y
    en fase activa. Se originan por reflejo vagal por compresión de la
    cabeza fetal.
  \item
    DIPS TIPO 2 (TARDIAS): comienzan después del inicio de la
    contracción su valor mas bajo se produce posterior al acmé de la
    contracción y retorna a su valor basal pasado el termino de la
    contracción. Refleja la presencia de hipoxia, sufrimiento fetal.
  \item
    DIPS TIPO 3 (VARIABLES): son desaceleraciones que varían mucho en
    amplitud y frecuencia, se relaciones con problemas de compresión de
    cordón y pueden traducir hipoxia fetal si son muy permanentes.
  \end{itemize}
\end{itemize}

{[}{[}vigilancia fetal electronica1.png{]}{]} \#\#\#\# Vigilancia fetal
ecográfica \#\#\#\#\# VELOCIMETRÍA DOPPLER: - Utilizada para valorar
flujo sanguíneo arterial y venoso más importantes de la madre (arterias
uterinas) y del feto (arterias cerebral, umbilical y ductos venosos). -
Rápido y no invasivo para evaluar el bienestar fetal y de los parámetros
hemodinpamicaos fetales. \#\#\#\# Vigilancia fetal bioquímica - DOSAJE
DE ESTRIOL: Valores menores a 4 mg en orina de 24 horas: Sufrimiento
fetal crónico - LACTOGENO PLACENTARIO: 32 semanas-dosaje por debajo de 4
ng/ml se relaciona en 75\% con sufrimiento fetal - ALFA FETO PROTEINA:
semana 16 a 18 - elevación traduce alteraciones del tubo neural. -
BIOQUIMICA SANGUINEA FETAL: PH, O2, CO2, (cuero cabelludo)

\end{document}
